\RequirePackage{ifxetex,ifluatex}
\newif\ifxetexorluatex
\ifxetex
  \xetexorluatextrue
\else
  \ifluatex
    \xetexorluatextrue
  \else
    \xetexorluatexfalse
  \fi
\fi

% Checks to see if correct compiler, then enables academicons
%% See texdoc.net/pkg/fontawecome and http://texdoc.net/pkg/academicons for full list of symbols. You MUST compile with XeLaTeX or LuaLaTeX if you want to use academicons.
\ifxetexorluatex
    \documentclass[12pt,letterpaper,ragged2e, academicons]{Resume}
    % "normalphoto" => Normal photo, not cropped
    % "ragged2e"    => Cropped photo to circle
\else
    \documentclass[12pt, letterpaper, ragged2e]{Resume}
\fi

% Change the page layout if you need to
% \geometry{left=1cm,right=9cm,marginparwidth=6.8cm,marginparsep=1.2cm,top=1.25cm,bottom=1.25cm}

% Change the font if you want to, depending on whether
% you're using pdflatex or xelatex/lualatex
\ifxetexorluatex
  % If using xelatex or lualatex:
  \setmainfont{Carlito}
\else
  % If using pdflatex:
  \usepackage[utf8]{inputenc}
  \usepackage[T1]{fontenc}
  \usepackage[default]{lato}
\fi

\usepackage{multicol}
\usepackage{parallel}
\usepackage[skins,breakable]{tcolorbox}
\usepackage{paracol}
\usepackage{parcolumns}

\raggedbottom

% Set the desired heading colors
\definecolor{headingColor}{HTML}{000000}
\definecolor{accentColor}{HTML}{ff0000}
\definecolor{emphasisColor}{HTML}{2E2E2E}
\definecolor{bodyColor}{HTML}{666666}

\colorlet{heading}{headingColor}
\colorlet{accent}{accentColor}
\colorlet{emphasis}{emphasisColor}
\colorlet{body}{bodyColor}

% Change the bullets for itemize and rating marker
% for \cvskill if you want to
\renewcommand{\itemmarker}{{\small\textbullet}}
\renewcommand{\ratingmarker}{\faCircle}

%%% sample.bib contains your publications
%\addbibresource{Publications.bib}

\begin{document}
    \name{Alex Tacescu}
    \phone{(559) 301-6222}
    \email{alextac98@gmail.com}
    \website{www.alextac.com} 
    %
    % Create Heading
    \makeheading
    %
    % Start body

    \resumesection{Qualification Summary}{}
Robotics engineer with extensive experience developing complex software simulations in robotics and aerospace industries and writing software for embedded platforms.
    \vspace{1mm}\\
    \resumesection{Technical Skills}{}
\skillsection{Robotics}{Software Development, Mechanical Design, Electrical Design, Agile Project Management (SCRUM)}\\
\skillsection{Programming}
{   C++ [6 years],                      % 2017
    Python [7 years],                   % 2016
    Robot Operating System [6 years],   % 2017
    Java [11 years],                     % 2012
    SQL [1 year],                       % Summer 2018
    PointCloudLibrary [2 years],        % Summer 2018
    Docker [5 years],                   % 2018
    and Git [9 years]                   % 2014
}\\
\skillsection{3D CAD}
{Design and Simulation in 
    Autodesk Inventor [11 years]         % 2012
    , Dassault SolidWorks [8 years]      % 2013-2021
    , PTC Creo Parametric (ProE) [1 year] % 2012-2013
}\\
\skillsection{Rapid Prototpying}{
    FDM \& SLA 3D Printers, Laser/WaterJet Cutting %pcb-design
}\\
\skillsection{Embedded Systems Programming}{Raspberry Pi, BeagleBone Black/Blue, NVIDIA Jetson, RTOS (ChibiOS) on STM32 and ESP32, Real-Time Linux Kernel with device trees}\\
\skillsection{Other Software Experience}{Linux (Debian, Ubuntu, RedHat, ArchLinux, etc), MATLAB, Adobe Creative Suite (Photoshop \& Premiere), KiCAD}
    \vspace{1mm}\\
    \resumesection{Education}{}
\school{M.S. in Computer Science}{Worcester Polytechnic Institute}{May 2020}{4.0}
    \vspace{-2mm}\\
    %% Provide the file name containing the sidebar contents as an optional parameter to \cvsection.
%% You can always just use \marginpar{...} if you do
%% not need to align the top of the contents to any
%% \cvsection title in the "main" bar.
\resumesection{Experience}

\cvevent{President \& CEO}{Yahoo!}{July 2012 -- Ongoing}{Sunnyvale, CA}
\begin{itemize}
\item Led the \$5 billion acquisition of the company with Verizon -- the entity which believed most in the immense value Yahoo! has created
\item Acquired Tumblr for \$1.1 billion and moved the company's blog there
\item Built Yahoo's mobile, video and social businesses from nothing in 2011 to \$1.6 billion in GAAP revenue in 2015
\item Tripled the company's mobile base to over 600 million monthly active users and generated over \$1 billion of mobile advertising revenue last year
\end{itemize}

\divider

\cvevent{Vice President of Location \& Services}{Google}{Oct 2010 -- July 2012}{Palo Alto, CA}
\begin{itemize}
\item Position Google Maps as the world leader in mobile apps and navigation
\item Oversaw 1000+ engineers and product managers working on Google Maps, Google Places and Google Earth
\end{itemize}

\divider

\cvevent{Vice President of Search Products \& UX}{Google}{2005 --  2010}{Palo Alto, CA}

\divider

\cvevent{Product Manager \& UI Lead}{Google}{Oct 2001 -- July 2005}{Palo Alto, CA}

\begin{itemize}
\item Appointed by the founder Larry Page in 2001 to lead the Product Management and User Interaction teams
\item Optimized Google's homepage and A/B tested every minor detail to increase usability (incl.~spacing between words, color schemes and pixel-by-pixel element alignment)
\end{itemize}

    \vspace{3mm}
    \resumesection{Projects}{For more, please visit \href{https://www.alextac.com/projects}{www.alextac.com/projects}}
\project{SmallKat Major Qualifying Project}{2018-Present}{is a quadrupedal robotic platform designed for research and development of multipedal robotic systems. SmallKat is 3D printed, open source, and contains fully custom electronics. I am developing the high-level software, including footstep planning, path planning, forward/inverse kinematics, machine vision, and networking systems for off-platform debugging for my Major Qualifying Project with 2 other students. To learn more, please visit \href{https://www.alextac.com/smallkat}{www.alextac.com/smallkat}}
\project{Project Maverick}{2015-Present}{is an award-winning omni-directional robotic system that provides mobility for people with walking disabilities. The drive system allows the user to move in any direction using 4 steering and 4 driving electronically synchronized motors, creating the same degrees of motion as an able person. It was designed, built, and programmed as a personal project, initially with Java and then converted to ROS (C++ \& Python). To learn more, please visit \href{https://www.pmaverick.com}{www.pmaverick.com}}
\project{Poverty Stoplight Interactive Qualifying Project}{2017-2018}{is an Android application for social workers in Paraguay to better help people in poverty. The application was designed for Fundación Paraguay and Poverty Stoplight and consisted of developing a REST API and an Android application capable of syncing sensitive family data with a secure server. To learn more, please visit \href{https://www.alextac.com/stoplight-iqp}{www.alextac.com/stoplight-iqp}}
\project{NASA Space Robotics Challenge}{2016-17}{is a competition to develop software for NASA’s humanoid robot Valkyrie. Developed footstep motion planning, optimized cycle-speed, and tested in ROS, C++, and Python with Gazebo as a member of the WPI Humanoid Robotics Lab. To learn more, please visit \href{https://www.alextac.com/src}{www.alextac.com/src}}
\project{Project Drogo}{2017}{is a wearable embedded system accompanied by a smart-phone app designed to assist elderly people through post-hip surgery recovery. It combines 2 goals of post-surgery medicine: preventing prohibited motions and guiding the user through physical therapy and rehab. Developed on a team of 4 students as a part of the hackathon Health Hacks RI, where it received 1st place. To learn more, please visit \href{https://www.alextac.com/drogo}{www.alextac.com/drogo}}
\project{Project Pather}{2017}{is a kiosk mapping software developed to provide directions to Bringham and Women’s Hospital visitors. It has contextual search as well as the capability to send directions to users via text message or email. It is written in Java and JavaFX, with a SQL backend, and was developed on an 8-person team for a school project. To learn more, please visit \href{https://www.alextac.com/pather}{www.alextac.com/pather}}
\project{FIRST FRC Robotics Team 2761}{2012-16}{4 cumulative seasons with the team. Designed, built, programmed, and tested 5 full-size robots. To learn more, please visit \href{https://www.alextac.com/frc}{www.alextac.com/frc}}
    \vspace{-2mm}\\
    \resumesection{Awards}{For an updated list, please visit \href{https://www.alextac.com/awards}{www.alextac.com/awards}}
	\resumesubsection{2018}
	\begin{itemize}
		\item Dean’s List at WPI (Spring 2018)
		\item Inducted in Rho Beta Epsilon Robotics Engineering Honor Society
	\end{itemize}
	%
	\resumesubsection{2017}
	\begin{itemize}
		\item Dean’s List at WPI (Fall 2017)
		\item 1st Place at HealthHacksRI at the University of Rhode Island for Project Drogo
		\item NASA Space Robotics Challenge Team Finalist
	\end{itemize}
	%
	\resumesubsection{2016}
	\begin{itemize}
		\item 2nd Place at the Intel International Science and Engineering Fair (ISEF) in the category of Applied Mechanics
		\item Google International Science Fair Regional Finalist
		\item International Council on Systems Engineering First Award for “best interdisciplinary project that can produce technologically appropriate solution that meet societal needs” at the ISEF
		\item GE Fallonventions Award and participation on NBC’s Tonight Show starring Jimmy Fallon (aired on April 11, 2016)
		\item Sweepstakes Award winner (1st place overall) and 1st place in Engineering at the Central California Science, Math, and Engineering Fair
		\item National Honor Society Inductee and California Scholarship Federation Member
	\end{itemize}
	%
	\resumesubsection{2015}
	\begin{itemize}
		\item Institute of Electrical and Electronics Engineers President’s Scholarship Award at Intel Science and Engineering Fair for “an outstanding project demonstrating an understanding of electrical engineering, electronics engineering, and computer science.”
		\item 1st place in the category of Applied Mechanics and Structures at the California State Science Fair
		\item Sweepstakes Award winner (1st place overall) and 1st place in Engineering at the Central California Science, Math, and Engineering Fair
	\end{itemize}
    \vspace{-3mm}\\
    \resumesection{Leadership Experience}{}
	\leadership{2019-2020}{Officer board member for the Rho Beta Epsilon Robotics Honor Society}
	\newline
	\leadership{2016}{Leadership Practice at WPI: analyzed business and leadership practices for an on-campus organization with Prof. Sharon Wulf}
	\newline
	\leadership{2012-16}{ Lead Technical Director, Build and Pit Team Leader for FIRST FRC Robotics Team 2761}
    \vspace{-3mm}\\
    \resumesection{Hobbies}{}
Making, Woodworking, Tennis, Ultimate Frisbee, Skiing, Fishing, Camping
    
\end{document}
