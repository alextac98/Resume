\resumesection{Projects}{For more, please visit \href{https://alextac.com/projects}{alextac.com/projects}}
\project{Project Maverick / MavDrive}{2015-Current}{is an award-winning omni-directional robotic system that provides mobility for people with walking disabilities. The drive system allows the user to move in any direction using 4 steering and 4 driving electronically synchronized motors, creating the same degrees of motion as an able person. It was designed, built, and programmed as a personal project, initially with Java and then converted to ROS (C++ \& Python). The drive system is currently being further developed with RTOS microcontrollers and distributed computing. To learn more, please visit \href{https://www.pmaverick.com}{pmaverick.com}}
\project{WPI Exoskeleton}{2020-2021}{is a research project with the goal to design an exoskeleton to help paraplegic patients' physical rehabilitation through exercise and bone loading. I developed a customizable bio-mechanical knee joint to follow the flexion and extension patterns of the human knee with the help of magnetic resonance imaging (MRI) as my Masters thesis. This project was done as a part of WPI's Automation and Interventional Medicine (AIM) robotics research laboratory. Published 3 papers, 2 as primary author.}
\project{SmallKat Major Qualifying Project}{2018-2019}{is a quadrupedal robotic platform designed for research and development of multipedal robotic systems. SmallKat is 3D printed, open source, and contains fully custom electronics. I developed the high-level software, including footstep planning, path planning, forward/inverse kinematics, machine vision, and networking systems for off-platform debugging for my Major Qualifying Project (MQP) with 2 other students. To learn more, please visit \href{https://alextac.com/smallkat}{alextac.com/smallkat}}
\project{Poverty Stoplight Interactive Qualifying Project}{2017-2018}{is an Android application for social workers in Paraguay to better help people in poverty. The application was designed for Fundación Paraguay and Poverty Stoplight and consisted of developing a REST API and an Android Java application capable of syncing sensitive family data with a secure server. To learn more, please visit \href{https://alextac.com/stoplight-iqp}{alextac.com/stoplight-iqp}}
\project{NASA Space Robotics Challenge}{2016-17}{is a competition to develop software for NASA’s humanoid robot Valkyrie. Developed footstep motion planning, optimized cycle-speed, and tested in ROS, C++, and Python with Gazebo as a member of the WPI Humanoid Robotics Lab. To learn more, please visit \href{https://alextac.com/src}{alextac.com/src}}
\project{Project Drogo}{2017}{is a wearable embedded system accompanied by a smart-phone app designed to assist elderly people through post-hip surgery recovery. It combines 2 goals of post-surgery medicine: preventing prohibited motions and guiding the user through physical therapy and rehab. Developed on a team of 4 students as a part of the hackathon Health Hacks RI, where it received 1st place. To learn more, please visit \href{https://alextac.com/drogo}{alextac.com/drogo}}
% \project{Project Pather}{2017}{is a kiosk mapping software developed to provide directions to Bringham and Women’s Hospital visitors. It has contextual search as well as the capability to send directions to users via text message or email. It is written in Java and JavaFX, with a SQL backend, and was developed on an 8-person team for a school project. To learn more, please visit \href{https://alextac.com/pather}{alextac.com/pather}}
\project{FIRST FRC Robotics Team 2761}{2012-16}{4 cumulative seasons with the team. Designed, built, programmed, and tested 5 full-size robots. To learn more, please visit \href{https://alextac.com/frc}{alextac.com/frc}}